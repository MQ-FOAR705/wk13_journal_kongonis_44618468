\documentclass{article}
\usepackage[utf8]{inputenc}

\title{wk13 journal}
\author{Thomas Kongonis}
\date{October 2019}

\begin{document}

\maketitle

\tableofcontents

\section{R Coding Modules}

\subsection{Module 1: Before We Start}
\begin{itemize}
\item{\textbf{Intention:} Complete activity '1'.}

\begin{itemize}
\item{\textbf{Action:} implemented install.packages for tidyverse.}
\item{\textbf{Result:} The instillation took a while but as it was outlined in the module, i saw the sideways arrow to show that it was completed.}
\end{itemize}
\end{itemize}


\subsection{Module 2: Introduction to R}

\begin{itemize}
\item{\textbf{Intention:} Complete activity '1'.}

\begin{itemize}
\item{\textbf{Action:} implemented the lines of code for both hectares and acres.}
\item{\textbf{Result:} The number for acres did not change at all. This is most likely due to the fact that its still operating under the initial code you ran.}
\end{itemize}

\item{\textbf{Intention:} Complete activity '2'.}

\begin{itemize}
\item{\textbf{Action:} in the script, i created a length and width variable and gave them values. After this i typed more lines where i changed the values of length and width.}
\item{\textbf{Result:} I proved what the exercise set out to do. I got 8 both times as the script ran through.}
\end{itemize}

\item{\textbf{Intention:} Complete activity '3'.}

\begin{itemize}
\item{\textbf{Action:}type in ?round.}
\item{\textbf{Result:} There are a few related functions that all concern what digits are shown of a number, you can round up and round down among other things. The digits function relates to the amount of digits that are shown.}
\end{itemize}

\item{\textbf{Intention:} Complete activity '4.1, 4.2, 4.3'.}

\begin{itemize}
\item{\textbf{Action:} Answering question.}
\item{\textbf{Result:} The program will convert them all into the same type.}

\item{\textbf{Action:} Answer question, implemented code.}
\item{\textbf{Result:} I think it relates to what was said in the last question. Because vectors can only contain one type of data, other types will be converted to the common data type.}

\item{\textbf{Action:} implement code and answer question.}
\item{\textbf{Result:} It looks like what happened is the top bit of code was changed to a one and this was reflected in the final bit of code. That being said I'm not sure whey this didn't happen for the second line also.}
\end{itemize}

\item{\textbf{Intention:} Complete activity '5'.}

\begin{itemize}
\item{\textbf{Action:} Implemented code script to first find the median of the rooms vector. made sure that the NA element was excluded with the is.na code and got the median 1. Then i created and assigned a variable for houses that used more than 2 rooms. This came from the rooms with no na section that had a input that's bigger than 2. This left us with 4 households that met this in the data set.}
\item{\textbf{Result:}.}
\end{itemize}



\end{itemize}



\subsection{Module 3: Starting with Data}

\begin{itemize}
 
\item{\textbf{Intention:} Complete activity '1'.}

\begin{itemize}
\item{\textbf{Action:}.}
\item{\textbf{Result:}.}
\end{itemize}


\item{\textbf{Intention:} Complete activity '2'.}

\begin{itemize}
\item{\textbf{Action:}.}
\item{\textbf{Result:}.}
\end{itemize}


\end{itemize}




\subsection{Module 4: Introducing dplyr and tidyr}

\begin{itemize}

\item{\textbf{Intention:} Complete activity '1'.}

\begin{itemize}
\item{\textbf{Action:}.}
\item{\textbf{Result:}.}
\end{itemize}

\item{\textbf{Intention:} Complete activity '2'.}

\begin{itemize}
\item{\textbf{Action:}.}
\item{\textbf{Result:}.}
\end{itemize}

\item{\textbf{Intention:} Complete activity '3.1, 3.2, 3.3'.}

\begin{itemize}
\item{\textbf{Action:}.}
\item{\textbf{Result:}.}

\item{\textbf{Action:}.}
\item{\textbf{Result:}.}

\item{\textbf{Action:}.}
\item{\textbf{Result:}.}


\end{itemize}

\item{\textbf{Intention:} Complete activity '4.1, 4.2'.}

\begin{itemize}
\item{\textbf{Action:}.}
\item{\textbf{Result:}.}

\item{\textbf{Action:}.}
\item{\textbf{Result:}.}


\end{itemize}




 
\end{itemize}

\section{Proof of Concept}

\subsection{Implementing User Stories}


\subsection{Results of Utilising the Shell Command Without GUI}


\subsection{Problems and Solutions/Quality Assurance}



\subsection{Data Recovery Plan/Version Control}




\section{PICO Presentation}

\subsection{Slide Creation}



\subsection{Scripting and Gathering Necessary Information}

\end{document}
