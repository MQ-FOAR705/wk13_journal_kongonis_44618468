\documentclass{article}
\usepackage[utf8]{inputenc}

\title{wk13 journal}
\author{Thomas Kongonis}
\date{October 2019}

\begin{document}

\maketitle

\tableofcontents

\section{R Coding Modules}

\subsection{Module 1: Before We Start}
\begin{itemize}
\item{\textbf{20/10/2019 9:00-9:30 Intention:} Complete activity '1'.}

\begin{itemize}
\item{\textbf{Action:} implemented install.packages for tidyverse.}
\item{\textbf{Result:} The instillation took a while but as it was outlined in the module, i saw the sideways arrow to show that it was completed.}
\end{itemize}
\end{itemize}


\subsection{Module 2: Introduction to R}

\begin{itemize}
\item{\textbf{20/10/2019 10:00-11:30 Intention:} Complete activity '1'.}

\begin{itemize}
\item{\textbf{Action:} implemented the lines of code for both hectares and acres.}
\item{\textbf{Result:} The number for acres did not change at all. This is most likely due to the fact that its still operating under the initial code you ran.}
\end{itemize}

\item{\textbf{Intention:} Complete activity '2'.}

\begin{itemize}
\item{\textbf{Action:} in the script, i created a length and width variable and gave them values. After this i typed more lines where i changed the values of length and width.}
\item{\textbf{Result:} I proved what the exercise set out to do. I got 8 both times as the script ran through.}
\end{itemize}

\item{\textbf{Intention:} Complete activity '3'.}

\begin{itemize}
\item{\textbf{Action:}type in ?round.}
\item{\textbf{Result:} There are a few related functions that all concern what digits are shown of a number, you can round up and round down among other things. The digits function relates to the amount of digits that are shown.}
\end{itemize}

\item{\textbf{Intention:} Complete activity '4.1, 4.2, 4.3'.}

\begin{itemize}
\item{\textbf{Action:} Answering question.}
\item{\textbf{Result:} The program will convert them all into the same type.}

\item{\textbf{Action:} Answer question, implemented code.}
\item{\textbf{Result:} I think it relates to what was said in the last question. Because vectors can only contain one type of data, other types will be converted to the common data type.}

\item{\textbf{Action:} implement code and answer question.}
\item{\textbf{Result:} It looks like what happened is the top bit of code was changed to a one and this was reflected in the final bit of code. That being said I'm not sure whey this didn't happen for the second line also.}
\end{itemize}

\item{\textbf{Intention:} Complete activity '5'.}

\begin{itemize}
\item{\textbf{Action:} Implemented code script to first find the median of the rooms vector. made sure that the NA element was excluded with the is.na code and got the median 1. Then i created and assigned a variable for houses that used more than 2 rooms. This came from the rooms with no na section that had a input that's bigger than 2. This left us with 4 households that met this in the data set.}
\item{\textbf{Result:}.}
\end{itemize}



\end{itemize}



\subsection{Module 3: Starting with Data}

\begin{itemize}
 
\item{\textbf{26/10/2019 16:00-17:00 Intention:} Complete activity '1'.}

\begin{itemize}
\item{\textbf{Action:} implemented the arrow interviews and within brackets 100. This returned no response so i did it again. After this i looked and within the data column in the environment section i had interviews100 there. For the second exercise, i tried to implement all of the codes but it was to difficult to do and nothing happened and the console sent back an error message. With the last 2 exercises, it seems as if i implemented the codes correctly, the two codes needed were quite similar.}
\item{\textbf{Result:}.}
\end{itemize}


\item{\textbf{Intention:} Complete activity '2'.}

\begin{itemize}
\item{\textbf{Action:} For the first part of this activity, really all that needed to be done was to utilise the same code that was used earlier in the lesson except make the first letters uppercase. This was implemented successfully. After this similar code needed to be used to rename the factor parts to be uppercase also. After this was all done, bringing up the plot showed that the letters had been successfully changed.}
\item{\textbf{Result:}.}
\end{itemize}


\end{itemize}




\subsection{Module 4: Introducing dplyr and tidyr}

\begin{itemize}

\item{\textbf{28/10/2019 12:00-14:00 Intention:} Complete activity '1'.}

\begin{itemize}
\item{\textbf{Action:} implemented the nested codes. First was interview then the nest code. second line was to filter the memb section with just the yes'. The final function was the select and in brackets just the things that were asked.}
\item{\textbf{Result:} This implementation was a success. Was rather straight forward.}
\end{itemize}

\item{\textbf{Intention:} Complete activity '2'.}

\begin{itemize}
\item{\textbf{Action:} take the interviews total meals column and rename it to just interviews with the arrow and then nest it. next is the mutate function with total meals and no members in brackets. Finally we do select and village and total meals.}
\item{\textbf{Result:}. The things in the environment section changed, so it would appear that it was a success but I'm not sure.}
\end{itemize}

\item{\textbf{Intention:} Complete activity '3.1, 3.2, 3.3'.}

\begin{itemize}
\item{\textbf{Action:} Implement code interviews, nest then count number of meals.}
\item{\textbf{Result:} 52 people had 2 meals and 79 people has 3 meals.}

\item{\textbf{Action:} Input interviews nest, group by village nest then summarise the average, minimum and max number of number of family members.}
\item{\textbf{Result:} Didn't get an error message and seemed to get a reasonable table come up so it appears to be successful.}

\item{\textbf{Action:} Needed to look at solution as i didn't know how to answer.}
\item{\textbf{Result:} I obviously missed something because i had no idea that i needed to load another library function. Still implemented though.}


\end{itemize}

\item{\textbf{Intention:} Complete activity '4.1, 4.2'.}

\begin{itemize}
\item{\textbf{Action:} Successfully created the months lack food column, nested and then implemented separate rows code for months lack food then separated with a semicolon like previous lesson showed. nested code the implemented a mutate code, nested then implemented spread.}
\item{\textbf{Result:} Everything seemed to work although i may have missed a thing here or there and am just unsure.}

\item{\textbf{Action:} selected the months lack food column.}
\item{\textbf{Result:} Couldn't figure out what else there was to do, didn't get any reasonable answer.}


\end{itemize}




 
\end{itemize}

\section{Proof of Concept}

\subsection{Implementing User Stories and Quality Assurance}

\begin{itemize}
\item{\textbf{User Story 1:} As a student, i want to be able to view multiple  translations alongside each other, so that i don’t have to go searching every time i get confused with a translated term.}
\begin{itemize}
\item{Implementing this user story hinged upon implementing all of the others effectively. Therefore, the steps towards attempting to implement this user story is the completion of the other user stories.} 
\item{10/11/2019 14:00 As of this time all testing and coding has been completed. Times were not effectively recorded due to negligence. That being said however the majority of work has been done sporadically over the past 2 weeks.}
\item{The gathering of translations occurred back in late September, these were transferred to chapters around the same time. Directories and work spaces were created late October and all of the coding and testing has been done over the past several days.}
\end{itemize}

\item{\textbf{User Story 2:} As a student i should be able to find public domain translations.}
\begin{itemize}
\item{This particular user story was implemented in late September/early October. There was some difficulty in this, initially i attempted to find 5 public domain translations that i had researched and identified before hand but i could not find 3 of them.}
\item{After going back to the drawing board, i needed to first and foremost fine new translations. Once these translations had been found, i looked for confirmation within the text itself that it was public domain, or if it was not that the fair use licence extended to personal/academic use. }
\item{Upon completion of this user story, i did not find the complete 5 translations as i had initially listed. Due to this i had to settle for four translations due to time constraints.}
\end{itemize}
\item{\textbf{User Story 3:} As a student, i should be able to use the Shell Command in lieu of a GUI.}
\begin{itemize}
\item{This part of the user story was addressed within section 2.2 of this document. Due to the scattered nature of my projects i made the error of implementing and recording some user stories in a notebook and some others on this document as i worked on them.} 
\end{itemize}
\item{\textbf{User Story 4:} As a student, i should be able to write a shell script that will accurately title each chapter of selected translations.}
\begin{itemize}
\item{Initial attempts at changing and moving files with the shell made it clear to me that my command over the content would not be enough to pull translations from entire documents or add coded titles within the large documents. Another unforeseen difficulty with this was that some translations had Chinese translations and other symbols that threw out the documents in a bad way. It would have taken too much time to take this out of the document.}
\item{The solution to this problem, was to take 5 chapters from each translation and move them into separate coded txt documents.}

\item{I attempted to do this with the shell and whilst being able to rename and move documents one at a time with a code i could not create a code to do it all at the same time. I did not have the time to re learn thee correct way to do it so the most time effective solution was to just utilise the simple mv code to move files and rename files.}
\end{itemize}
\item{\textbf{User Story 5:} As a student, i should be able to write a shell script that can pull up a specific chapter of 4 translations within the shell command.}
\begin{itemize}
\item{10/11/2019 13:00 Simple codes were written to do this that aligned with the directory names.} 
\end{itemize}
    
\end{itemize}


\subsection{Results of Utilising the Shell Command Without GUI}

\begin{itemize}
    \item{\textbf{ 29/10/2019 11:49 Changing file names:} I have been utilising the unix shell to copy and move files and create directories for my data set to apply my shell script to. I continually tried to implement wildcards to change the names of my .txt documents. After multiple failures i decided that it would be quicker to individually change each file within the shell rather than spending 3 hours looking and learning the right code. Due to this, i successfully implemented name changes and directory making. I ended up creating a separate directory for each chapter and had a separate .txt for each translated chapter in the format t01c01.txt for example. I implemented the code mv t01c01 t01c02 for example. After a mistake i realised i needed to add the .txt and had success replicating this for each required file name change.} 
\end{itemize}


\subsection{Problems and Solutions/Quality Assurance}

\begin{itemize}
    \item{\textbf{29/10/2019 13:00 .xml:} After downloading the relevant translations and converting them to relevant .txt documents, i tried to convert them to .xml through using notepad2. I had a success with t01 and i recorded exactly what i did. Following this, even though i attempted to do the exact same thing i could not replicate what i did and i could not find out why. As i could not create correct .xml files, i decided it would be a better avenue to alter my project to utilising individual .txt documents with each translated chapter in separate directories per chapter and then utilising a wildcard code to bring the relevant chapters up in the shell command.}
    
    \item{\textbf{Writing Shell Script:} In testing there were some difficulties namely formatting documents correctly, changing folders into a uniform coded name format and also writing scripts that would work across different directories. }
    \item{I would test by writing and testing codes within the console from simple until more complex and i got a desired result and then i would rewrite these into a shell script and test until i got the correct response. The main problems in this were the difference in directories which i remedied by adding cd codes before the cat commands.}
    \item{As much of the things i set out were to difficult for me, I went back to the drawing board and created simple scripts that worked, through a basic application of wildcards i was able to get the desired response, although i would add that this software deployment at this stage is more of an exercise in quantity over quality. The main point to combat this however is that the scripts work!}
    
   
\end{itemize}



\subsection{Data Recovery Plan/Version Control}

\begin{itemize}
\item{09/11/2019 12:40 My data recovery plan that was implemented, included back ups onto usb documents, cloudstor, google drive and a github repository. Due to the time sensitive nature of other projects, this was all implemented rather haphazardly and in a very confusing way. The only backups that were clear and easy to find were the documents saved to the overleaf servers. I could always rely on them being there and furthermore their connection to github meant that i had a way to diagnose and resolve problems as needed. }
\end{itemize}

\section{PICO Presentation}

\subsection{Slide Creation}

\begin{itemize}
\item{10/11/2019 16:00 Due to other assessments and time constraints, i have yet to do anymore work than a brief scripting in a workbook.} 
\end{itemize}



\subsection{Scripting and Gathering Necessary Information}

\section{Concluding thoughts}

\begin{itemize}
\item{I would say that the R coding was probably a bit too difficult for me and i struggled. What i really enjoyed however was finishing up the software deployment. It was a nice change from essay writing and its a satisfying feeling to have your brainchild not break a computer even if it is a bit simplisitic.} 
\end{itemize}

\end{document}
